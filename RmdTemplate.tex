\documentclass[$if(fontsize)$$fontsize$,$endif$$if(lang)$$babel-lang$,$endif$$if(papersize)$$papersize$paper,$endif$$for(classoption)$$classoption$$sep$,$endfor$]{$documentclass$}
$if(fontfamily)$
\usepackage[$for(fontfamilyoptions)$$fontfamilyoptions$$sep$,$endfor$]{$fontfamily$}
$else$
\usepackage{lmodern}
$endif$
$if(linestretch)$
\usepackage{setspace}
\setstretch{$linestretch$}
$endif$
\usepackage{amssymb,amsmath,mathptmx,stmaryrd}
\usepackage{ifxetex,ifluatex}
\usepackage{fixltx2e} % provides \textsubscript
\ifnum 0\ifxetex 1\fi\ifluatex 1\fi=0 % if pdftex
  \usepackage[$if(fontenc)$$fontenc$$else$T1$endif$]{fontenc}
  \usepackage[utf8]{inputenc}
$if(euro)$
  \usepackage{eurosym}
$endif$
\else % if luatex or xelatex
  \ifxetex
    \usepackage{mathspec}
  \else
    \usepackage{fontspec}
  \fi
  \defaultfontfeatures{Ligatures=TeX,Scale=MatchLowercase}
$if(euro)$
  \newcommand{\euro}{€}
$endif$
$if(mainfont)$
    \setmainfont[$for(mainfontoptions)$$mainfontoptions$$sep$,$endfor$]{$mainfont$}
$endif$
$if(sansfont)$
    \setsansfont[$for(sansfontoptions)$$sansfontoptions$$sep$,$endfor$]{$sansfont$}
$endif$
$if(monofont)$
    \setmonofont[Mapping=tex-ansi$if(monofontoptions)$,$for(monofontoptions)$$monofontoptions$$sep$,$endfor$$endif$]{$monofont$}
$endif$
$if(mathfont)$
    \setmathfont(Digits,Latin,Greek)[$for(mathfontoptions)$$mathfontoptions$$sep$,$endfor$]{$mathfont$}
$endif$
$if(CJKmainfont)$
    \usepackage{xeCJK}
    \setCJKmainfont[$for(CJKoptions)$$CJKoptions$$sep$,$endfor$]{$CJKmainfont$}
$endif$
\fi
% use upquote if available, for straight quotes in verbatim environments
\IfFileExists{upquote.sty}{\usepackage{upquote}}{}
% use microtype if available
\IfFileExists{microtype.sty}{%
\usepackage{microtype}
\UseMicrotypeSet[protrusion]{basicmath} % disable protrusion for tt fonts
}{}
$if(geometry)$
\usepackage[$for(geometry)$$geometry$$sep$,$endfor$]{geometry}
$endif$
\usepackage{hyperref}
$if(colorlinks)$
\PassOptionsToPackage{usenames,dvipsnames}{color} % color is loaded by hyperref
$endif$
\hypersetup{unicode=true,
$if(title-meta)$
            pdftitle={$title-meta$},
$endif$
$if(author-meta)$
            pdfauthor={$author-meta$},
$endif$
$if(keywords)$
            pdfkeywords={$for(keywords)$$keywords$$sep$; $endfor$},
$endif$
$if(colorlinks)$
            colorlinks=true,
            linkcolor=$if(linkcolor)$$linkcolor$$else$Maroon$endif$,
            citecolor=$if(citecolor)$$citecolor$$else$Blue$endif$,
            urlcolor=$if(urlcolor)$$urlcolor$$else$Blue$endif$,
$else$
            pdfborder={0 0 0},
$endif$
            breaklinks=true}
\urlstyle{same}  % don't use monospace font for urls
$if(lang)$
\ifnum 0\ifxetex 1\fi\ifluatex 1\fi=0 % if pdftex
  \usepackage[shorthands=off,$for(babel-otherlangs)$$babel-otherlangs$,$endfor$main=$babel-lang$]{babel}
$if(babel-newcommands)$
  $babel-newcommands$
$endif$
\else
  \usepackage{polyglossia}
  \setmainlanguage[$polyglossia-lang.options$]{$polyglossia-lang.name$}
$for(polyglossia-otherlangs)$
  \setotherlanguage[$polyglossia-otherlangs.options$]{$polyglossia-otherlangs.name$}
$endfor$
\fi
$endif$
$if(natbib)$
\usepackage{natbib}
\bibliographystyle{$if(biblio-style)$$biblio-style$$else$plainnat$endif$}
$endif$
$if(biblatex)$
\usepackage$if(biblio-style)$[style=$biblio-style$]$endif${biblatex}
$if(biblatexoptions)$\ExecuteBibliographyOptions{$for(biblatexoptions)$$biblatexoptions$$sep$,$endfor$}$endif$
$for(bibliography)$
\addbibresource{$bibliography$}
$endfor$
$endif$
$if(listings)$
\usepackage{listings}
$endif$
$if(lhs)$
\lstnewenvironment{code}{\lstset{language=Haskell,basicstyle=\small\ttfamily}}{}
$endif$
$if(highlighting-macros)$
$highlighting-macros$
$endif$
$if(verbatim-in-note)$
\usepackage{fancyvrb}
\VerbatimFootnotes % allows verbatim text in footnotes
$endif$
$if(tables)$
\usepackage{longtable,booktabs,widetable}
$endif$

% always, graphics silly
$if(graphics)$
\usepackage{graphicx,grffile}
\makeatletter
\def\maxwidth{\ifdim\Gin@nat@width>\linewidth\linewidth\else\Gin@nat@width\fi}
\def\maxheight{\ifdim\Gin@nat@height>\textheight\textheight\else\Gin@nat@height\fi}
\makeatother
% Scale images if necessary, so that they will not overflow the page
% margins by default, and it is still possible to overwrite the defaults
% using explicit options in \includegraphics[width, height, ...]{}
\setkeys{Gin}{width=\maxwidth,height=\maxheight,keepaspectratio}
$endif$

%%%%%%%%%%%%%%%%%%%%%%%%%%%%%%%%%%%%%%%%%%%%%%%%%%%%
% Taken from bookdown preamble
%%%%%%%%%%%%%%%%%%%%%%%%%%%%%%%%%%%%%%%%%%%%%%%%%%%%

\usepackage{booktabs}
\usepackage{longtable}
\usepackage{framed,color}
\definecolor{shadecolor}{RGB}{248,248,248}

\ifxetex
  \usepackage{letltxmacro}
  \setlength{\XeTeXLinkMargin}{1pt}
  \LetLtxMacro\SavedIncludeGraphics\includegraphics
  \def\includegraphics#1#{% #1 catches optional stuff (star/opt. arg.)
    \IncludeGraphicsAux{#1}%
  }%
  \newcommand*{\IncludeGraphicsAux}[2]{%
    \XeTeXLinkBox{%
      \SavedIncludeGraphics#1{#2}%
    }%
  }%
\fi

\newenvironment{rmdblock}[1]
  {\begin{shaded*}
  \begin{itemize}
  \renewcommand{\labelitemi}{
    \raisebox{-.7\height}[0pt][0pt]{
      {\setkeys{Gin}{width=3em,keepaspectratio}\includegraphics{images/#1}}
    }
  }
  \item
  }
  {
  \end{itemize}
  \end{shaded*}
  }
\newenvironment{rmdnote}
  {\begin{rmdblock}{note}}
  {\end{rmdblock}}
\newenvironment{rmdcaution}
  {\begin{rmdblock}{caution}}
  {\end{rmdblock}}
\newenvironment{rmdimportant}
  {\begin{rmdblock}{important}}
  {\end{rmdblock}}
\newenvironment{rmdtip}
  {\begin{rmdblock}{tip}}
  {\end{rmdblock}}
\newenvironment{rmdwarning}
  {\begin{rmdblock}{warning}}
  {\end{rmdblock}}

%%%%%%%%%%%%%%%%%%%%%%%%%%%%%%%%%%%%%%%%%%%%%%%%%%%%
% Taken from bookdown preamble
%%%%%%%%%%%%%%%%%%%%%%%%%%%%%%%%%%%%%%%%%%%%%%%%%%%%


$if(links-as-notes)$
% Make links footnotes instead of hotlinks:
\renewcommand{\href}[2]{#2\footnote{\url{#1}}}
$endif$
$if(strikeout)$
\usepackage[normalem]{ulem}
% avoid problems with \sout in headers with hyperref:
\pdfstringdefDisableCommands{\renewcommand{\sout}{}}
$endif$
$if(indent)$
$else$
\IfFileExists{parskip.sty}{%
\usepackage{parskip}
}{% else
\setlength{\parindent}{0pt}
\setlength{\parskip}{6pt plus 2pt minus 1pt}
}
$endif$
\setlength{\emergencystretch}{3em}  % prevent overfull lines
\providecommand{\tightlist}{%
  \setlength{\itemsep}{0pt}\setlength{\parskip}{0pt}}
$if(numbersections)$
\setcounter{secnumdepth}{5}
$else$
\setcounter{secnumdepth}{0}
$endif$
$if(subparagraph)$
$else$
% Redefines (sub)paragraphs to behave more like sections
\ifx\paragraph\undefined\else
\let\oldparagraph\paragraph
\renewcommand{\paragraph}[1]{\oldparagraph{#1}\mbox{}}
\fi
\ifx\subparagraph\undefined\else
\let\oldsubparagraph\subparagraph
\renewcommand{\subparagraph}[1]{\oldsubparagraph{#1}\mbox{}}
\fi
$endif$
$if(dir)$
\ifxetex
  % load bidi as late as possible as it modifies e.g. graphicx
  $if(latex-dir-rtl)$
  \usepackage[RTLdocument]{bidi}
  $else$
  \usepackage{bidi}
  $endif$
\fi
\ifnum 0\ifxetex 1\fi\ifluatex 1\fi=0 % if pdftex
  \TeXXeTstate=1
  \newcommand{\RL}[1]{\beginR #1\endR}
  \newcommand{\LR}[1]{\beginL #1\endL}
  \newenvironment{RTL}{\beginR}{\endR}
  \newenvironment{LTR}{\beginL}{\endL}
\fi
$endif$

%%% Use protect on footnotes to avoid problems with footnotes in titles
\let\rmarkdownfootnote\footnote%
\def\footnote{\protect\rmarkdownfootnote}

%%% Change title format to be more compact
\usepackage{titling}

% Create subtitle command for use in maketitle
\newcommand{\subtitle}[1]{
 \posttitle{
   \begin{center}\large#1\end{center}
   }
}

\setlength{\droptitle}{-2em}
$if(title)$
 \title{$title$}
 \pretitle{\vspace{\droptitle}\centering\huge}
 \posttitle{\par}
$else$
 \title{}
 \pretitle{\vspace{\droptitle}}
 \posttitle{}
$endif$
$if(subtitle)$
\subtitle{$subtitle$}
$endif$
$if(author)$
 \author{$for(author)$$author$$sep$ \\ $endfor$}
 \preauthor{\centering\large\emph}
 \postauthor{\par}
$else$
 \author{}
 \preauthor{}\postauthor{}
$endif$
$if(date)$
 \predate{\centering\large\emph}
 \postdate{\par}
 \date{$date$}
$else$
 \date{}
 \predate{}\postdate{}
$endif$


$for(header-includes)$
$header-includes$
$endfor$

\begin{document}

$if(title)$
\maketitle
$endif$
$if(abstract)$
\begin{abstract}
$abstract$
\end{abstract}
$endif$

$for(include-before)$
$include-before$

$endfor$
$if(toc)$
{
$if(colorlinks)$
\hypersetup{linkcolor=$if(toccolor)$$toccolor$$else$black$endif$}
$endif$
\setcounter{tocdepth}{$toc-depth$}
\tableofcontents
}
$endif$
$if(lot)$
\addcontentsline{toc}{chapter}{LIST OF TABLES}
\listoftables
$endif$
$if(lof)$
\cleardoublepage \phantomsection \addcontentsline{toc}{chapter}{LIST OF FIGURES}
\listoffigures
$endif$

\cleardoublepage \phantomsection
\specialchapt{ACKNOWLEDGEMENTS}

This thesis would not be possible without many people. First and foremost, special thanks to my major professor Heike Hofmann. I would not made it to this point without such a warm and friendly mentor who was often more confident in my abilities than I was of my own. Thank you for always supporting me no matter how many things I had going on to distract me from research. I aspire to inherit the same empathy and support that you show to your students on a daily basis. 

Another special thanks goes to Di Cook. In the fourth year of my PhD, I was experiencing burnout (and growing tired of Ames), when Di took me to lunch, told me she was transferring to Monash University in Australia, and invited me to join her. Of course, I said yes, and when I arrived, I immediately felt welcomed and a part of the group -- all thanks to Di. She strategically assigned me to assist her students with their thesis projects, run R workshops for the university, and most importantly, work on my tennis game. Through this "work" I met so many amazing people and had many memorable experiences. Those 6 months gave me a new perspective on life in general and I am forever grateful for being blessed enough to take the opportunity.

Thank you to many of Heike and Di's former students who came before me (just to name a few: Hadley Wickham, Michael Lawrence, Yihui Xie, Xiaoyue Cheng, Barrett Schloerke, Susan VanderPlas). Your work has not only inspired and enabled my work, but it has also enabled an entire community of people working with data to do amazing things. Without this strong history and community at Iowa State, I would not have had the courage or the vision to follow such a "non-traditional" research path. I hope the University continues to value this type of work as it teaches students skills that are in high demand and generally improves the way data-driven research is performed.

Thank you to all my collaborators, especially Toby Dylan Hocking. Toby and Susan VanderPlas laid the initial framework for \textbf{animint} -- which I first worked on as a Google Summer of Code student under Toby's guidance. Toby later went on write the initial version of the \texttt{ggplotly}() function in \textbf{plotly}, borrowing a lot of ideas from \textbf{animint}. As Toby became busy with other things, he introduced me to the plotly team, and eventually handed over the reigns on the project, which has helped to financially support the last year or so of grad school.

Thank you also to the plotly team, and in particular, the software engineers who work on the open source project plotly.js. My work has benefited greatly from your responsiveness to my questions, feature requests, and bug reports. I have a great amount of respect for the work that you do, and I hope this project keeps improving at its current break neck pace.

Finally, thank you to my family for their encouragement and keeping me grounded throughout this experience. Thank you to my father for the initial encouragement to pursue a PhD, conversations surrounding work-life balance, and also pushing me to "graduate before I'm 40". Thank you to my mother for her unconditional love, endless care, and predenting to understand what I do for a living. Thank you to my brothers for providing me with shelter, beer, and Twins tickets. Thank you all for your willingness to drop everything to help me at any given moment. I can't say I've always been as willing, as I have been selfish with my time during my PhD, but I hope to change that after graduation.


\newpage
\pagenumbering{arabic}
$body$


% for some reason this seems to break lof/lot? 

%$if(natbib)$
%$if(bibliography)$
%$if(biblio-title)$
%$if(book-class)$
%\renewcommand\bibname{$biblio-title$}
%$else$
%\renewcommand\refname{$biblio-title$}
%$endif$
%$endif$
%\bibliography{$for(bibliography)$$bibliography$$sep$,$endfor$}
%
%$endif$
%$endif$
%$if(biblatex)$
%\printbibliography$if(biblio-title)$[title=$biblio-title$]$endif$

$endif$
$for(include-after)$
$include-after$

$endfor$

\end{document}